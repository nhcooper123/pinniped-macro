% Table of biogeography data
\begin{longtable}{llccccccccccp{0.6\textwidth}}

\caption{Biogeography data table for all 120 taxa included in the biogeographical history analyses. (A) Northern Pacific Ocean, (B) Eastern South Pacific Ocean, (C) Australasian seas and coastal waters, (D) Northern Atlantic Ocean, (E) Southern Atlantic Ocean, (F) Indian Ocean, (G) Southern Ocean, (H) Arctic Ocean, and (I) Paratethys/Mediterranean Ocean. Status defines the taxa as Extant or Fossil. Total gives the total number of areas each taxon is present within.}\\

\hline
\textbf{Taxon} & \textbf{Status} & \textbf{A} & \textbf{B} & \textbf{C} & \textbf{D} & \textbf{E} & \textbf{F}
& \textbf{G} & \textbf{H} & \textbf{I} & \textbf{Total} & \textbf{Notes}\\
\hline 
\textit{Arctocephalus australis} &
Extant  &
0 &
1 &
0 &
0 &
1 &
0 &
0 &
0 &
0 &
2  &
\\

\textit{Arctocephalus forsteri} &
Extant  &
0 &
0 &
1 &
0 &
0 &
1 &
0 &
0 &
0 &
2  &
\\

\textit{Arctocephalus galapagoensis} &
Extant &
0 &
1 &
0 &
0 &
0 &
0 &
0 &
0 &
0 &
1 &
\\

\textit{Arctocephalus gazella} &
Extant &
0 &
0 &
0 &
0 &
0 &
0 &
1 &
0 &
0 &
1 &
\\

\textit{Arctocephalus philippii} &
Extant &
0 &
1 &
0 &
0 &
0 &
0 &
0 &
0 &
0 &
1 &
\\

\textit{Arctocephalus pusillus} &
Extant &
0 &
0 &
1 &
0 &
1 &
1 &
0 &
0 &
0 &
3 &
\\

\textit{Arctocephalus townsendi} &
Extant &
1 &
0 &
0 &
0 &
0 &
0 &
0 &
0 &
0 &
1 &
\\

\textit{Arctocephalus tropicalis} &
Extant &
0 &
0 &
0 &
0 &
1 &
1 &
1 &
0 &
0 &
3 &
\\

\textit{Callorhinus ursinus} &
Extant &
1 &
0 &
0 &
0 &
0 &
0 &
0 &
0 &
0 &
1 &
\\

\textit{Cystophora cristata} &
Extant &
0 &
0 &
0 &
1 &
0 &
0 &
0 &
1 &
0 &
2 &
\\

\textit{Erignathus barbatus} &
Extant &
1 &
0 &
0 &
1 &
0 &
0 &
0 &
1 &
0 &
3 &
\\

\textit{Eumetopias jubatus} &
Extant &
1 &
0 &
0 &
0 &
0 &
0 &
0 &
0 &
0 &
1 &
\\

\textit{Halichoerus grypus} &
Extant &
0 &
0 &
0 &
1 &
0 &
0 &
0 &
1 &
0 &
2 &
\\

\textit{Histriophoca fasciata} &
Extant &
1 &
0 &
0 &
0 &
0 &
0 &
0 &
1 &
0 &
2 &
\\

\textit{Hydrurga leptonyx} &
Extant &
0 &
0 &
0 &
0 &
0 &
0 &
1 &
0 &
0 &
1 &
\\

\textit{Leptonychotes weddellii} &
Extant &
0 &
0 &
0 &
0 &
0 &
0 &
1 &
0 &
0 &
1 &
\\

\textit{Lobodon carcinophaga} &
Extant &
0 &
0 &
0 &
0 &
0 & 
0 &
1 &
0 &
0 &
1 &
\\

\textit{Mirounga angustirostris} &
Extant &
1 &
0 &
0 &
0 &
0 &
0 &
0 &
0 &
0 &
1 &
\\

\textit{Mirounga leonina} &
Extant &
0 &
1 &
1 &
0 &
0 &
0 &
1 &
0 &
0 &
3 &
\\

\textit{Monachus monachus} &
Extant &
0 &
0 &
0 &
1 &
0 &
0 &
0 &
0 &
1 &
2 &
\\

\textit{Neomonachus schauinslandi} &
Extant &
1 &
0 &
0 &
0 &
0 &
0 &
0 &
0 &
0 &
1 &
\\


\textit{Neomonachus tropicalis} &
Extant &
0 &
0 &
0 &
1 &
0 &
0 &
0 &
0 &
0 &
1 &
\\

\textit{Neophoca cinerea} &
Extant &
0 &
0 &
1 &
0 &
0 &
1 &
0 &
0 &
0 &
2 &
\\

\textit{Odobenus rosmarus} &
Extant &
1 &
0 &
0 &
0 &
0 &
0 &
0 &
1 &
0 &
2 &
Bering Sea counts as North Pacific\\

\textit{Ommatophoca rossii} &
Extant &
0 &
0 &
0 &
0 &
0 &
0 &
1 &
0 &
0 &
1 &
\\

\textit{Otaria byronia} &
Extant &
0 &
1 &
0 &
0 &
1 &
0 &
0 &
0 &
0 &
2 &
\\

\textit{Pagophilus groenlandica} &
Extant &
0 &
0 &
0 &
1 &
0 &
0 &
0 &
1 &
0 &
2 &
\\

\textit{Phoca largha} &
Extant &
1 &
0 &
0 &
0 &
0 &
0 &
0 &
1 &
0 &
2 &
Bering Sea counts as North Pacific\\

\textit{Phoca vitulina} &
Extant &
1 &
0 &
0 &
1 &
0 &
0 &
0 &
1 &
0 &
3 &
\\

\textit{Phocarctos hookeri} &
Extant &
0 &
0 &
1 &
0 &
0 &
0 &
0 &
0 &
0 &
1 &
\\

\textit{Pusa caspica} &
Extant &
0 &
0 &
0 &
0 &
0 &
0 &
0 &
0 &
1 &
1 &
Caspian sea was part of Parathethys\\

\textit{Pusa hispida} &
Extant &
1 &
0 &
0 &
1 &
0 &
0 &
0 &
1 &
0 &
3 &
\\

\textit{Zalophus californianus} &
Extant &
1 &
0 &
0 &
0 &
0 &
0 &
0 &
0 &
0 &
1 &
\\

\textit{Zalophus japonicus} &
Extant &
1 &
0 &
0 &
0 &
0 &
0 &
0 &
0 &
0 &
1 &
\\

\textit{Zalophus wollebaeki} &
Extant &
0 &
1 &
0 &
0 &
0 &
0 &
0 &
0 &
0 &
1 &
\\

\textit{Acrophoca longirostris} &
Fossil &
0 &
1 &
0 &
0 &
0 &
0 &
0 &
0 &
0 &
1 &
\\

\textit{Aivukus cedrosensis} &
Fossil &
1 &
0 &
0 &
0 &
0 &
0 &
0 &
0 &
0 &
1 &
\\

\textit{Allodesmus demerei} &
Fossil &
1 &
0 &
0 &
0 &
0 &
0 &
0 &
0 &
0 &
1 &
\\

\textit{Allodesmus kernensis} &
Fossil &
1 &
0 &
0 &
0 &
0 &
0 &
0 &
0 &
0 &
1 &
\\

\textit{Allodesmus naorai} &
Fossil &
1 &
0 &
0 &
0 &
0 &
0 &
0 &
0 &
0 &
1 &
\\

\textit{Allodesmus packardi} &
Fossil &
1 &
0 &
0 & 
0 &
0 &
0 &
0 &
0 &
0 &
1 &
\\

\textit{Allodesmus sinanoensis} &
Fossil &
1 &
0 &
0 &
0 &
0 &
0 &
0 &
0 &
0 &
1 &
\\

\textit{Allodesmus uraiporensis} &
Fossil &
1 &
0 &
0 &
0 &
0 &
0 &
0 &
0 &
0 &
1 &
\\

\textit{Allodesmus sp cf sadoensis} &
Fossil &
1 &
0 &
0 &
0 &
0 &
0 &
0 &
0 &
0 &
0 &
\\

\textit{Archaeodobenus akamatsui} &
Fossil &
1 &
0 &
0 &
0 &
0 &
0 &
0 &
0 &
0 &
1 &
\\

\textit{Atopotarus courseni} &
Fossil &
1 &
0 &
0 &
0 &
0 &
0 &
0 &
0 &
0 &
0 &
\\

\textit{Australophoca changorum} &
Fossil &
0 &
1 &
0 &
0 &
0 &
0 &
0 &
0 &
0 &
1 &
\\

\textit{Callorhinus gilmorei} &
Fossil &
1 &
0 &
0 &
0 &
0 &
0 &
0 &
0 &
0 &
1 &
\\

\textit{Desmatophoca brachycephala} &
Fossil &
1 &
0 &
0 &
0 &
0 &
0 &
0 &
0 &
0 &
1 &
\\

\textit{Desmatophoca oregonensis} &
Fossil &
1 &
0 &
0 &
0 &
0 &
0 &
0 &
0 &
0 &
1 &
\\

Desmatophocidae indet &
Fossil &
1 &
0 &
0 &
0 &
0 &
0 &
0 &
0 &
0 &
0 &
Desmatophocidae indet USNM 335445\\

\textit{Devinophoca claytoni} &
Fossil &
0 &
0 &
0 &
0 &
0 &
0 &
0 &
0 &
1 &
1 &
\\

\textit{Devinophoca emryi} &
Fossil &
0 &
0 &
0 &
0 &
0 &
0 &
0 &
0 &
1 &
1 &
\\

\textit{Dusignathus santacruzensis} &
Fossil &
1 &
0 &
0 &
0 &
0 &
0 &
0 &
0 &
0 &
1 &
\\

\textit{Dusignathus seftoni} &
Fossil &
1 &
0 &
0 &
0 &
0 &
0 &
0 &
0 &
0 &
1 &
\\

\textit{Enaliarctos barnesi} &
Fossil &
1 &
0 &
0 &
0 &
0 &
0 &
0 &
0 &
0 &
1 &
\\

\textit{Enaliarctos emlongi} &
Fossil &
1 &
0 &
0 &
0 &
0 &
0 &
0 &
0 &
0 &
1 &
\\

\textit{Enaliarctos mealsi} &
Fossil &
1 &
0 &
0 &
0 &
0 &
0 &
0 &
0 &
0 &
1 &
\\

\textit{Enaliarctos mitchelli} &
Fossil &
1 &
0 &
0 &
0 &
0 &
0 &
0 &
0 &
0 &
1 &
\\

\textit{Enaliarctos tedfordi} &
Fossil &
1 &
0 &
0 &
0 &
0 &
0 &
0 &
0 &
0 &
1 &
\\

\textit{Eodesmus condoni} &
Fossil &
1 &
0 &
0 &
0 &
0 &
0 &
0 &
0 &
0 &
1 &
\\

\textit{Eomonachus belegaerensis} &
Fossil &
0 &
0 &
1 &
0 &
0 &
0 &
0 &
0 &
0 &
1 &
\\

\textit{Eotaria citrica} &
Fossil &
1 &
0 &
0 &
0 &
0 &
0 &
0 &
0 &
0 &
1 &
\\

\textit{Eotaria crypta} &
Fossil &
1 &
0 &
0 &
0 &
0 &
0 &
0 &
0 &
0 &
1 &
\\

\textit{Frisiphoca aberratum} &
Fossil &
0 &
0 &
0 &
1 &
0 &
0 &
0 &
0 &
0 &
1 &
\\

\textit{Frisiphoca affine} &
Fossil &
0 &
0 &
0 &
1 &
0 &
0 &
0 &
0 &
0 &
1 &
\\

\textit{Gomphotaria pugnax} &
Fossil &
1 &
0 &
0 &
0 &
0 &
0 &
0 &
0 &
0 &
1 &
\\

\textit{Hadrokirus martini} &
Fossil &
0 &
1 &
0 &
0 &
0 &
0 &
0 &
0 &
0 &
1 &
\\

\textit{Homiphoca capensis} &
Fossil &
0 &
0 &
0 &
0 &
1 &
0 &
0 &
0 &
0 &
1 &
Not coded as North Atlantic as Spanish material is dubious\\

aff \textit{Homiphoca capensis} &
Fossil &
0 &
0 &
0 &
1 &
0 &
0 &
0 &
0 &
0 &
0 &
Phocidae aff \textit{Homiphoca capensis} USNM\\

\textit{Homiphoca sp} &
Fossil &
0 &
0 &
0 &
0 &
1 &
0 &
0 &
0 &
0 &
0 &
Combined \textit{Homiphoca sp} SAM PQL 30080, 32415, 31976, 32101, 30568 as one taxon\\

\textit{Hydrarctos lomasiensis} &
Fossil &
0 &
1 &
0 &
0 &
0 &
0 &
0 &
0 &
0 &
1 &
\\

\textit{Imagotaria downsi} &
Fossil &
1 &
0 &
0 &
0 &
0 &
0 &
0 &
0 &
0 &
1 &
\\

\textit{Kamtschatarctos sinelnikovae} &
Fossil &
1 &
0 &
0 &
0 &
0 &
0 &
0 &
0 &
0 &
1 &
\\

\textit{Kawas benegasorum} &
Fossil &
0 &
0 &
0 &
0 &
1 &
0 &
0 &
0 &
0 &
1 &
\\

\textit{Leptophoca proxima} &
Fossil &
0 &
0 &
0 &
1 &
0 &
0 &
0 &
0 &
1 &
2 &
\\

cf Miroungini indet &
Fossil &
0 &
0 &
1 &
0 &
0 &
0 &
0 &
0 &
0 &
0 &
cf Miroungini indet CD 35\\

Monachini indet &
Fossil &
0 &
0 &
1 &
0 &
0 &
0 &
0 &
0 &
0 &
0 &
Monachini indet NMV P160399\\

\textit{Nanophoca vitulinoides} &
Fossil &
0 &
0 &
0 &
1 &
0 &
0 &
0 &
0 &
0 &
1 &
\\

\textit{Neophoca palatina} &
Fossil &
0 &
0 &
1 &
0 &
0 &
0 &
0 &
0 &
0 &
1 &
\\

\textit{Neotherium mirum} &
Fossil &
1 &
0 &
0 &
0 &
0 &
0 &
0 &
0 &
0 &
1 &
\\

\textit{Noriphoca gaudini} &
Fossil &
0 &
0 &
0 &
0 &
0 &
0 &
0 & 
0 &
1 &
1 &
\\

Odobenidae indet &
Fossil &
1 &
0 &
0 &
0 &
0 &
0 &
0 &
0 &
0 &
0 &
Odobenidae gen et spec indet LACM 135920\\

\textit{Ontocetus emmonsi} &
Fossil &
0 &
0 &
0 &
1 &
0 &
0 &
0 &
0 &
0 &
1 &
\\

\textit{Osodobenus eodon} &
Fossil &
1 &
0 &
0 &
0 &
0 &
0 &
0 &
0 &
0 &
1 &
\\

\textit{Pelagiarctos thomasi} &
Fossil &
1 &
0 &
0 &
0 &
0 &
0 &
0 &
0 &
0 &
1 &
\\

\textit{Pelagiarctos sp} &
Fossil &
1 &
0 &
0 &
0 &
0 &
0 &
0 &
0 &
0 &
0 &
\textit{Pelagiarctos sp} SDNHM 131041\\

\textit{Pinnarctidion bishopi} &
Fossil &
1 &
0 &
0 &
0 &
0 &
0 &
0 &
0 &
0 &
1 &
\\

\textit{Pinnarctidion iverseni} &
Fossil &
1 &
0 &
0 &
0 &
0 &
0 &
0 &
0 &
0 &
0 &
\\

\textit{Pinnarctidion rayi} &
Fossil &
1 &
0 &
0 &
0 &
0 &
0 &
0 &
0 &
0 &
1 &
\\

\textit{Pithanotaria starri} &
Fossil &
1 &
0 &
0 &
0 &
0 &
0 &
0 &
0 &
0 &
0 &
\\

\textit{Pliopedia pacifica} &
Fossil &
1 &
0 &
0 &
0 &
0 &
0 &
0 &
0 &
0 &
1 &
\\

\textit{Pliophoca etrusca} &
Fossil &
0 &
0 &
0 &
0 &
0 &
0 &
0 &
0 &
1 &
1 &
\\

\textit{Pliophoca etrusca USNM} &
Fossil &
0 &
0 &
0 &
1 &
0 &
0 &
0 &
0 &
0 &
0 &
\textit{Pliophoca etrusca} USNM 171221 et 181419 et 181504 et 187580 et 243697 et 250290 et 254327 et 34734\\

\textit{Pontolis barroni} &
Fossil &
1 &
0 &
0 &
0 &
0 &
0 &
0 &
0 &
0 &
1 &
\\

\textit{Pontolis} cf &
Fossil &
1 &
0 &
0 &
0 &
0 &
0 &
0 &
0 &
0 &
0 &
\\

\textit{Pontolis kohnoi} &
Fossil &
1 &
0 &
0 &
0 &
0 &
0 &
0 &
0 &
0 &
1 &
\\

\textit{Pontolis magnus} &
Fossil &
1 &
0 &
0 &
0 &
0 &
0 &
0 &
0 &
0 &
1 &
\\

\textit{Potamotherium vallentoni} &
Fossil &
0 &
0 &
0 &
1 &
0 &
0 &
0 &
0 &
0 &
1 &
\\

\textit{Praepusa boeska} &
Fossil &
0 &
0 &
0 &
0 &
0 &
0 &
0 &
0 &
1 &
1 &
\\

\textit{Praepusa magyaricus} &
Fossil &
0 &
0 &
0 &
0 &
0 &
0 &
0 &
0 &
1 &
1 &
\\

\textit{Praepusa pannonica} &
Fossil &
0 &
0 &
0 &
1 &
0 &
0 &
0 &
0 &
1 &
2 &
\\

\textit{Praepusa vindobonensis} &
Fossil &
0 &
0 &
0 &
0 &
0 &
0 &
0 &
0 &
1 &
1 &
\\

\textit{Proneotherium repenningi} &
Fossil &
1 &
0 &
0 &
0 &
0 &
0 &
0 &
0 &
0 &
1 &
\\

\textit{Properiptychus argentinus} &
Fossil &
0 &
0 &
0 &
0 &
1 &
0 &
0 &
0 &
0 &
1 &
\\

\textit{Proterozetes ulysses} &
Fossil &
1 &
0 &
0 &
0 &
0 &
0 &
0 &
0 &
0 &
1 &
\\

\textit{Protodobenus japonicus} &
Fossil &
1 &
0 &
0 &
0 &
0 &
0 &
0 &
0 &
0 &
1 &
\\

\textit{Prototaria planicephala} &
Fossil &
1 &
0 &
0 &
0 &
0 &
0 &
0 &
0 &
0 &
1 &
\\

\textit{Prototaria primigena} &
Fossil &
1 &
0 &
0 &
0 &
0 &
0 &
0 &
0 &
0 &
1 &
\\

\textit{Pseudotaria muramotoi} &
Fossil &
1 &
0 &
0 &
0 &
0 &
0 &
0 &
0 &
0 &
1 &
\\

\textit{Piscophoca pacifica} &
Fossil &
0 &
1 &
0 &
0 &
0 &
0 &
0 &
0 &
0 &
1 &
\\

\textit{Pteronarctos goedertae} &
Fossil &
1 &
0 &
0 &
0 &
0 &
0 &
0 &
0 &
0 &
1 &
\\

\textit{Puijila darwini} &
Fossil &
0 &
0 &
0 &
0 &
0 &
0 &
0 &
1 &
0 &
1 &
\\

\textit{Sarcodectes magnus} &
Fossil &
0 &
0 &
0 &
1 &
0 &
0 &
0 &
0 &
0 &
1 &
\\

cf \textit{Thalassoleon}&
Fossil &
1 &
0 &
0 &
0 &
0 &
0 &
0 &
0 &
0 &
0 &
\\

\textit{Thalassoleon inouei} &
Fossil &
1 &
0 &
0 &
0 &
0 &
0 &
0 &
0 &
0 &
1 &
\\

\textit{Thalassoleon macnallyae} &
Fossil &
1 &
0 &
0 &
0 &
0 &
0 &
0 &
0 &
0 &
1 &
\\

\textit{Thalassoleon mexicanus} &
Fossil &
1 &
0 &
0 &
0 &
0 &
0 &
0 &
0 &
0 &
1 &
\\

\textit{Titanotaria orangensis} &
Fossil &
1 &
0 &
0 &
0 &
0 &
0 &
0 &
0 &
0 &
1 &
\\

\textit{Valenictus chulavistensis} &
Fossil &
1 &
0 &
0 &
0 &
0 &
0 &
0 &
0 &
0 &
1 &
\\
\hline

\label{table-biogeo-data}
\end{longtable}